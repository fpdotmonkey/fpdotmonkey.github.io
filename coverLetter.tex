%!TEX TS-program = xelatex
%!TEX encoding = UTF-8 Unicode

%
% coverLetter.tex
%
% Copyright (c) 2019 Fletcher Porter
%

\documentclass[12pt,american,german,british]{letter}
\usepackage{geometry}
\geometry{
	letterpaper,
	top=0.7in,
	left=0.8in,
	right=0.8in,
	bottom=0.6in
}

\usepackage{hyperref}
\usepackage{fontspec,xltxtra,xunicode}
\usepackage{babel}
\usepackage[iso,inputamerican]{isodate}

\setromanfont[]{Bona Nova}


\signature{Fletcher Porter \\ me@fletcherporter.com \\ +1 (626) 321-6687}

\begin{document}
\raggedright

\begin{letter}{NASA Jet Propulsion Laboratory}

\opening{Hello,}

My name is Fletcher Porter.  I'm a software developer and I'm interested in
your Scientific Applications Software Engineer I position.

I recently worked at Oblong Inc. developing appliance software for their
Linux-based Mezzanine system.  My work included qualifying new hardware for the
system to run on, audio programming in Python with PulseAudio,
troubleshooting hardware audio issues on Cisco codecs, maintaining Oblong's API
serialization and RPC libraries in C++ and Python, front-end and back-end
programming for the system's administration tools in Go, Ruby, Javascript, and
HTML, configuring the system's Nginx HTTP server, and video programming with
Gstreamer in C.  Throughout my work, I improved unit test coverage, inline
documentation, workflow documentation, Docker configurations, Continuous
Integration (CI) pipelines, and code quality so the next developer would have
an easier time than me.

Prior to that, I worked with a volunteer group responding to the COVID-19
pandemic to develop an open-source ventilator splitter.  The work I did was
much the same as at Oblong but with the addition of embedded Linux programming
to read data off of GPIO temperature and pressure sensors on a Raspberry Pi.  I
also took on many administrative tasks like become the Board Secretary,
applying for and receiving IRS 501(c)(3) non-profit status, and applying for a
bank account.

On my resume you'll see that my education is in the realm of hardware.  Design, 
CAD, simulations, machine tools, and assembly is all familiar fare to me.

I think I'd be very well-suited to this position at JPL.  My largest project at
Oblong was to qualify new hardware for Mezzanine to run on.  This involved
verifying and validating that every system in Mezzanine worked on the new
hardware.  In the case of the audio system, I needed to study behaviour down to
the wire level for troubleshooting.  I also have experience with space-based
instrumentation from JPL (347B) where I helped to design a prototype salt-
sampling instrument for an expidition to Death Valley.  As well, I have
experience with Matlab from school; scripting languages are very familiar to
me.

I also have strong communication skills which I've practiced in formal
presentations, reports, and papers for the work I have done, a few of which are
available to view in my portfolio, available at http://fletcherporter.com/.

Thank you for your consideration.  I hope to hear from you soon.


\closing{Cheers,}

\end{letter}

\end{document}























