%!TEX TS-program = xelatex
%!TEX encoding = UTF-8 Unicode

%
% Fletcher_Porter.tex
%
% Copyright (c) 2019 Fletcher Porter
%

\documentclass[12pt, oneside]{article}
\usepackage[margin = 0.70in]{geometry} \geometry{a4paper} 
\usepackage{graphicx}
\usepackage{amssymb}
\usepackage{color,soul}
\usepackage{fontspec,xltxtra,xunicode}
\usepackage{enumitem}
\usepackage{hyperref}
\usepackage[per-mode=symbol]{siunitx}
\usepackage{url}
\usepackage{etoolbox}
\usepackage{siunitx}
\patchcmd{\thebibliography}{\section*{\refname}}{}{}{}

\pagenumbering{gobble}


%%%%%%%%%%%%%%%%%%%%%%%%
% Formatting Functions %
%%%%%%%%%%%%%%%%%%%%%%%%

\makeatletter
\renewcommand\@biblabel[1]{}
\makeatother

\definecolor{new_red}{RGB}{94, 7, 3}

\defaultfontfeatures{Mapping=tex-text}
\setromanfont[]{Bona Nova}
\setsansfont[]{PT Sans Narrow Bold}

\newcommand{\titlestyle}[1] {
	{\fontsize{40pt}{1em}\selectfont \textcolor{new_red}{\textsf{#1}}} \\
}

\newcommand{\headingstyleJobs}[1] {
	{\fontsize{18pt}{1em}\selectfont \textcolor{new_red}{\textsf{#1}}}
	\textcolor{new_red}{\rule{3.25in}{0.5pt}} \vspace{3pt}
}

\newcommand{\infostyle}[1] {
	{\selectfont #1} \\ %\vspace{10pt}
}

\newcommand{\redemph}[1] {
	\hspace{-3pt}\textcolor{new_red}{#1}\hspace{-3pt}
}

\newcommand{\jobtitle}[3] {
	{\bf #1} · {#2} · {#3} \vspace{-7pt} \\
}

\newcommand{\certification}[2] {
	{\bf #1} · {#2}
}

\renewenvironment{quote}{%
  \list{}{%
    \leftmargin4pt
    \rightmargin\leftmargin
  }
  \item\relax
}
{\endlist}
\newcommand{\overview}[1] {
	\begin{quote}
		#1
	\end{quote}
}

\let\olditemize\itemize
\renewcommand{\itemize}{
  \olditemize
  \setlength{\itemsep}{0pt}
  \setlength{\parskip}{0pt}
  \setlength{\parsep}{0pt}
% \setlength{\topsep}{0pt}
}
\renewcommand{\labelitemi}{--}
\setlist[itemize]{leftmargin = 17pt}

\hypersetup{
  colorlinks,
  urlcolor=new_red
}
\urlstyle{same}

\begin{document}

\begin{flushleft}


%%%%%%%%%%%
% Content %
%%%%%%%%%%%

\titlestyle{Fletcher Porter}
\infostyle{+358 41 476 3167 · \href{mailto:me@fletcherporter.com}{me@fletcherporter.com} · \url{http://fletcherporter.com}}

\overview{
Minä olen moniosaava insinööri jolla on konesuunnittelun ja ohjelmistokehityksen erityistaitoja.  Minä tykkään osallistua järjestelmän.  Minä löydän ilon kaiken laajassa hahmottamisessa ja yksityiskoihtien tarkemmassa ymmärtämisessä.
}

\headingstyleJobs{Oppinot}

\jobtitle{Aalto-yliopisto}{Konetekniikan maisteri}{9/2022–}
\begin{itemize}
    \item Erikoisala mekatronikassa
\end{itemize}

\jobtitle{University of California, Santa Barbara (UCSB)}{BS Mechanical Engineering}{9/2015–9/2019}
\begin{itemize}
	\item Vaihto-opinnot 2017–2018 Lunds Tekniska Högskolassa \\
%	\item Courses in robotic control at UCSB including physics-based modeling \\
%	\item Kurssit robotikass, design with polymer composite materials, batteries, and fuel cells
\end{itemize}

%\jobtitle{Monrovia High School}{High School Diploma}{Graduated June 2015}
%\vspace{1em}

\certification{Certified SolidWorks Associate}{1/2020} \vspace{5pt}

\headingstyleJobs{Kielit}

\certification{Englanti}{Äidinkieli} \certification{Suomi}{A2} \vspace{5pt}

\headingstyleJobs{Kokemus}

\jobtitle{Aalto Yliopisto Fluid Power Lab}{Tutkimusapulainen}{10/2022–}
\begin{itemize}
	\item Kehittelin ohjaimesta järjestelmästä teollisuudenhydrauliseille Beckhoff TwinCATilla \\
	\item Yhdistyin CAN ja EtherCAT väylät yksitäiseksi järjestelmäksin \\
	\item Kehittelin viestiyhteyskestä ulko-ohlejelmiston kanssa TCP socketeilla \\
	\item Suunitelen akseli yhdistyä ABB:ista motorrista teollisuudenhydrauliseen järjestelmään
\end{itemize}

\jobtitle{Oblong Inc.}{Ohjelmiston Kehittäjä}{12/2020–3/2022}
\begin{itemize}
	\item Valmistuin uuden laitteiston monimutkainen, multimedia, laitteisto/ohjelmisto laite \\
	%\item Collaborated on a hybrid remote/in-person team in the midst of the COVID-19 pandemic \\
	%\item Improved logging on products to quicken Oblong's response to customer issues \\
	\item Etsiin ja poistin ääni virheet Linuxin, PulseAudion, ja Oblongin laitteiston kanssa \\
	%\item Took on maintainance and development of Oblong's legacy products and  software tools \\
	\item Loin ohjelmistoa Pythonilla, Golla, Bashilla, C++:illa, Rubylla, JavaScriptilla, ja TypeScriptilla \\
\end{itemize}

\jobtitle{NASA Jet Propulsion Laboratory}{Robottikka Harjoittelija}{Kesät 2018, 2016, ja 2014}
\begin{itemize}
	\item Kehittelin järjestelmä opiskella robottiliikkuvuus avaruuslennoille jääsiin kuihin \\
	\item Suunnittelin CAD:illa koneen purkaa painovoima vähän lateraalisein voimien kanssa \\
	\item Valmistin tekniset piirustukset koneelliseista komponenteista lähettää myyjien tehdä \\
	\item Suunnittelin sähköinen järjestelmä jakaa teho mikroprosessoreihin jota ohjaile järjestelmä \\
%	\item Assembled the mechanical and electrical systems by hand \\
	\item Kirjoitin itsetoimivaohjaimein ohjelmistoin ja dokumentaatio Pythonilla ja Arduino C:illa \\
	\item Kehittelin ehdotus \href{https://portfolium.com/entry/owms-deep-subsurface-access-level-wind}{avaruusluotain porata \textasciitilde$\SI{20}{\kilo\meter}$ Jupiterin kuun Europan jäänäiseen puoriin} \\
	%\item Designed and executed a test to demonstrate the feasibility of the proposed system \\
%	\item Created drawings for parts to manufacture and to get ROM quotes \\
	\item Yhteiskirjoittin ``\href{https://www.researchgate.net/publication/317702124_A_deep_subsurface_ice_probe_for_Europa}{A deep subsurface ice probe for Europa}'' IEEE Aerospace Conference 2017 \\
\end{itemize}

\jobtitle{Tetra Bio Distributed}{Ohjelmiston Kehittäjä}{2/2020–1/2021}
\begin{itemize}
	\item Kehittelin \href{https://github.com/tetrabiodistributed/project-tetra-display}{ohjelmisto näyttö} näyttää hengitys tieto COVID-19 potilaksesta lääkäreihin \\
	%\item Designed a web server to send patient data to a browser front end using Python and Go \\
	%\item Developed a Continuous Integration system to automatically test the software as it's checked in \\
	\item Kirjoitin ajurit paineelle ja virtalle antureille sulauttetulle Linuxille järjestelmä \\
	\item Loin signaalikäsitelly työkalut muuttaa anturitieton potilastiedoksi lääkäreille \\
%	\item Prepared IRS Form 1023-EZ to successfully get the organization 501(c)(3) non-profit status \\
\end{itemize}

\jobtitle{Hawkes Group, UCSB}{Tutkimusapulainen}{3/2019–9/2019}
\begin{itemize}
	\item Suunnittelin kiintokaluste pitää työkalusta \href{https://portfolium.com/entry/vine-robot-tool-mount}{pehmeä, köynnöskasvin kaltainen robottin} päällä \\
	\item Kehittelin mallin hyppykorkeudesta hiilikuitunainen jousi hyppyvä robotti \\
	\item Rakensin nämä kiintokalusteet 3D tulostimesta  \\
%	\item Documented the design process and results \\
\end{itemize}

\jobtitle{Kandidaatin Tutkielma, UCSB}{Insinööri opiskelija}{9/2018–6/2019}
\begin{itemize}
	\item Suunnittelin SolidWorksilla \href{https://portfolium.com/entry/automatic-stair-climbing-vehicle}{portaat-kiipeävä kärry} jota kantaa painavat hyötykuormat \\
	\item Tein useimmat komponentit jyrsimella ja sorvilla piirustukseista jota tein \\
%	\item Reported the design process and results \\
\end{itemize}

%\jobtitle{NASA Jet Propulsion Laboratory}{Robotics Intern}{June 2016 – Aug. 2016}
%\begin{itemize}
%	\item Developed a proposal for \href{https://portfolium.com/entry/owms-deep-subsurface-access-level-wind}{a probe to bore \textasciitilde$\SI{20}{\kilo\meter}$ into Europa's icy crust} \\
%	\item Designed and executed a test to demonstrate the feasibility of the proposed system \\
%	\item Created drawings for parts to manufacture and to get ROM quotes \\
%	\item Coauthored ``\href{https://ieeexplore-ieee-org.proxy.library.ucsb.edu:9443/stamp/stamp.jsp?tp=&arnumber=7943863}{A deep subsurface ice probe for Europa}'' in IEEE Aerospace Conference 2017 \\
%\end{itemize}

%\jobtitle{Mechatronics Course Design Project (LTH)}{Student Engineer}{Aug. 2017 – May 2018}
%\begin{itemize}
%	\item Designed and built \href{https://portfolium.com/entry/automatic-security-camera-lens-cleaner}{a mechanical system for cleaning security cameras}
%	\item Documented the design process and results in a report
%\end{itemize}

%\jobtitle{NASA Jet Propulsion Laboratory}{Robotics Intern}{June 2014 – Aug. 2014}
%\begin{itemize}
%	\item Designed components of \href{https://portfolium.com/entry/rescue-robots}{a robot manipulator} in CAD and iterated on the design
%%	\item Worked in the UCSB robotics lab with Katie Byl
%%	\item Developed roller skates for RoboSimian to significantly increase locomotive speed
%\end{itemize}

%\jobtitle{Robotic Planning and Kinematics Class}{UCSB ME 179P}{Jan. 2019 – March 2019}
%\begin{itemize}
%	\item Wrote \href{https://github.com/fpdotmonkey/ME179P}{planning and kinematics programs} in Python
%	\item Documented code with Python docstrings
%\end{itemize}

%\jobtitle{Robotic Controls Class}{UCSB ME 179D}{Sept. 2018 – Dec. 2018}
%\begin{itemize}
%	\item Wrote control programs for robot arms in Matlab and Simulink
%\end{itemize}

%\jobtitle{CNC Router PCB Fixture}{Personal Project}{2019-10-16 – 2019-11-08}
%\begin{itemize}
%	\item Designed a fixture in OnShape to hold a PCB while it's being machined
%	\item Created manufacturing drawings of all custom components
%	\item Machined components on a CNC router
%\end{itemize}

%\jobtitle{FIRST Tech Challenge}{Student Robotics Mentor}{June 2015–Present}
%\begin{itemize}
%	\item Teaching high school students engineering design
%\end{itemize}

%\jobtitle{Finite Element Method Course Projects}{}{Aug. 2017–Present}
%\begin{itemize}
%	\item Writing FEM simulation programs in Matlab using the CALFEM package
%\end{itemize}

%\jobtitle{Computer Music Generation System}{}{Sept.–April 2014}
%\begin{itemize}
%	\item Wrote a C++ program that produced audio tones by sending signals to the computer's sound card
%\end{itemize}


%\headingstyleJobs{Publications}
%
%\vspace{-9pt}
%\nocite{OWMS}
%\bibliography{bananas}
%\bibliographystyle{ieeetr}


\textit{Työnäytekansio minun työstäni löytyy osoitteesta} \url{http://portfolium.com/fporter/portfolio}

\end{flushleft}
\end{document}  